%%%%%%%%%%%%%%%%%%%%%%%%%%%%%%%%%%%%%%%%%
% Thesis Configuration File
%
% The main lines to change in this file are in the DOCUMENT VARIABLES
% section, the rest of the file is for advanced configuration.
%
%%%%%%%%%%%%%%%%%%%%%%%%%%%%%%%%%%%%%%%%%

%----------------------------------------------------------------------------------------
%	DOCUMENT VARIABLES
%	Fill in the lines below to enter your information into the thesis template
%	Each of the commands can be cited anywhere in the thesis
%----------------------------------------------------------------------------------------

% Remove drafting to get rid of the '[ Date - classicthesis version 4.0 ]' text at the bottom of every page
%\PassOptionsToPackage{eulerchapternumbers,listings,drafting, pdfspacing, subfig,beramono,eulermath,parts}{classicthesis}
\PassOptionsToPackage{eulerchapternumbers,listings,pdfspacing,dottedtoc,floatperchapter}{classicthesis}

% Available options: drafting parts nochapters linedheaders eulerchapternumbers beramono eulermath pdfspacing minionprospacing tocaligned dottedtoc manychapters listings floatperchapter subfig
% Adding 'dottedtoc' will make page numbers in the table of contents flushed right with dots leading to them

\newcommand{\myTitle}{Stories within immersive virtual environments\xspace}
%\newcommand{\mySubtitle}{ }
\newcommand{\myDegree}{ (Dr.-Ing.)\xspace}
\newcommand{\myName}{Joan Llobera\xspace}
\newcommand{\myProf}{Mel Slater\xspace}
%\newcommand{\myOtherProf}{Put name here\xspace}
%\newcommand{\mySupervisor}{Put name here\xspace}
%\newcommand{\myFaculty}{Put data here\xspace}
%\newcommand{\myDepartment}{Put data here\xspace}
\newcommand{\myUni}{ Universitat de Barcelona \xspace}
%\newcommand{\myLocation}{Darmstadt\xspace}
\newcommand{\myTime}{October 2012\xspace}
\newcommand{\myVersion}{version 4.0\xspace}

%----------------------------------------------------------------------------------------
%	USEFUL COMMANDS
%----------------------------------------------------------------------------------------

\newcommand{\ie}{i.\,e.}
\newcommand{\Ie}{I.\,e.}
\newcommand{\eg}{e.\,g.}
\newcommand{\Eg}{E.\,g.} 

\newcounter{dummy} % Necessary for correct hyperlinks (to index, bib, etc.)
\providecommand{\mLyX}{L\kern-.1667em\lower.25em\hbox{Y}\kern-.125emX\@}

%----------------------------------------------------------------------------------------
%	PACKAGES
%----------------------------------------------------------------------------------------

%\usepackage{lipsum} % Used for inserting dummy 'Lorem ipsum' text into the template

%------------------------------------------------
 
%\PassOptionsToPackage{latin9}{inputenc} % latin9 (ISO-8859-9) = latin1+"Euro sign"
%\usepackage{inputenc}
 
\usepackage[utf8x]{inputenc} 
\usepackage[british]{babel}

 %------------------------------------------------

%\PassOptionsToPackage{ngerman,american}{babel}  % Change this to your language(s)
% Spanish languages need extra options in order to work with this template
%\PassOptionsToPackage{spanish,es-lcroman}{babel}
\usepackage{babel}


 
 %------------------------------------------------

\PassOptionsToPackage{fleqn}{amsmath} % Math environments and more by the AMS 
 \usepackage{amsmath}
 

% Additional mathematics support
%\usepackage{amsmath}    % For numering of equations in chapters,
\usepackage{amssymb}

%IT DOES NOT WORK 
% Numbering of equations and figures based on chapter structure
%\numberwithin{equation}{chapter}
%\numberwithin{figure}{chapter}


%bold math symbols
\usepackage{bm} %use with \pmb

% check this: 
% $\underset{p.o.}{\underline{[ a \Rightarrow_{(\leq)} b] \equiv [a \wedge \neg_{(\leq)} a \ \leq \ b ]}}$

\newcommand{\po}[1]
{
\underset{p.o.}{\underline{\pmb{ #1 }}}
}



%define equations by parts:

\newcommand{\twopartdef}[4]
{
	\left\{
		\begin{array}{ll}
			#1 & \mbox{if } #2 \\
			#3 & \mbox{if } #4
		\end{array}
	\right.
}

%%example: the absolute value function is defined as: |x| = \twopartdef { x } {x \geq 0} {-x} {x < 0}




 %------------------------------------------------

\PassOptionsToPackage{T1}{fontenc} % T2A for cyrillics
\usepackage{fontenc}

%------------------------------------------------

\usepackage{xspace} % To get the spacing after macros right

%------------------------------------------------

\usepackage{mparhack} % To get marginpar right

%------------------------------------------------

\usepackage{fixltx2e} % Fixes some LaTeX stuff 

%------------------------------------------------

\PassOptionsToPackage{smaller}{acronym} % Include printonlyused in the first bracket to only show acronyms used in the text
\usepackage{acronym} % nice macros for handling all acronyms in the thesis

%------------------------------------------------

%\renewcommand*{\acsfont}[1]{\textssc{#1}} % For MinionPro
%\renewcommand{\bflabel}[1]{{#1}\hfill} % Fix the list of acronyms

%------------------------------------------------

\PassOptionsToPackage{final,pdftex}{graphicx}
\usepackage{graphicx} 


% Graphics inclusion
\usepackage[final,pdftex]{graphicx}
%\usepackage[tight]{subfigure}


% Color support - dvips color names (see section 7 in Graphics
% and Color in Latex)
%\usepackage[dvips]{color}
%\input{dvipsnam.def}


%%solve mess with page numbers:
\usepackage{nopageno}%http://www.tex.ac.uk/cgi-bin/texfaq2html?label=nopageno


%
% Indexing facilities
%
%\usepackage{index}


%
% Acronyms/glossaries typesetting
%
%\usepackage{nomencl}
\usepackage{gloss}
%
% Bibliography
%
% Natural sciences cite author year etc.
%\PassOptionsToPackage{square,numbers}{natbib}
\usepackage{natbib}{\markboth{}{}}
%\renewenvironment{thebibliography}[1]%
 %    {\protect{\thispagestyle{empty}}\pagestyle{empty}%
      %\markboth{}{}%
%}

%INCOMPATIBLE:
%\usepackage{biblatex}
%\bibsetup{\thispagestyle{empty}}


% Numerical reference style, see natnotes.pdf for more info
%\bibpunct{[}{]}{;}{n}{}{,}

%\bibliographystyle{plainnat}
\bibliographystyle{authordate2}



%%To get References instead of Bibliography
%\addto\captionsenglish{\renewcommand{\bibname}{References}}
%\addto\captionsenglish{\renewcommand{\refname}{References}}

%\renewcommand{\refname}{References}  % for the article class
%\renewcommand{\bibname}{References}  % for the report or book class

\AtBeginDocument{\renewcommand{\bibname}{References}}


%----------------------------------------------------------------------------------------
%	FLOATS: TABLES, FIGURES AND CAPTIONS SETUP
%----------------------------------------------------------------------------------------

\usepackage{tabularx} % Better tables
\setlength{\extrarowheight}{3pt} % Increase table row height
\newcommand{\tableheadline}[1]{\multicolumn{1}{c}{\spacedlowsmallcaps{#1}}}
\newcommand{\myfloatalign}{\centering} % To be used with each float for alignment
\usepackage{caption}
\captionsetup{format=hang,font=small}
\usepackage{subfig}  

%----------------------------------------------------------------------------------------
%	CODE LISTINGS SETUP
%----------------------------------------------------------------------------------------

%\usepackage{listings} 
%%\lstset{emph={trueIndex,root},emphstyle=\color{BlueViolet}}%\underbar} % for special keywords
%\lstset{language=[LaTeX]Tex, % Specify the language for listings here
%keywordstyle=\color{RoyalBlue}, % Add \bfseries for bold
%basicstyle=\small\ttfamily, % Makes listings a smaller font size and a different font
%%identifierstyle=\color{NavyBlue}, % Color of text inside brackets
%commentstyle=\color{Green}\ttfamily, % Color of comments
%stringstyle=\rmfamily, % Font type to use for strings
%numbers=left, % Change left to none to remove line numbers
%numberstyle=\scriptsize, % Font size of the line numbers
%stepnumber=5, % Increment of line numbers
%numbersep=8pt, % Distance of line numbers from code listing
%showstringspaces=false, % Sets whether spaces in strings should appear underlined
%breaklines=true, % Force the code to stay in the confines of the listing box
%%frameround=ftff, % Uncomment for rounded frame
%frame=single, % Frame border - none/leftline/topline/bottomline/lines/single/shadowbox/L
%belowcaptionskip=.75\baselineskip % Space after the "Listing #: Desciption" text and the listing box
%}



%%better code chunks that in verbatim:
%\usepackage{alltt}
\usepackage{listings}


\lstset{ %
language=C++,                % choose the language of the code
basicstyle=\footnotesize,       % the size of the fonts that are used for the code
numbers=left,                   % where to put the line-numbers
numberstyle=\footnotesize,      % the size of the fonts that are used for the line-numbers
stepnumber=2,                   % the step between two line-numbers. If it's 1 each line 
                                % will be numbered
numbersep=5pt,                  % how far the line-numbers are from the code
% backgroundcolor=\color{white},  % choose the background color. You must add \usepackage{color}
showspaces=false,               % show spaces adding particular underscores
showstringspaces=false,         % underline spaces within strings
showtabs=false,                 % show tabs within strings adding particular underscores
frame=single,	                % adds a frame around the code
tabsize=5,	                % sets default tabsize to 2 spaces
captionpos=b,                   % sets the caption-position to bottom
breaklines=true,                % sets automatic line breaking
breakatwhitespace=false,        % sets if automatic breaks should only happen at whitespace
%title=\lstname,                 % show the filename of files included with \lstinputlisting;
                                % also try caption instead of title
%escapeinside={\%*}{*)},         % if you want to add a comment within your code
%morekeywords={*,...}            % if you want to add more keywords to the set
}




%----------------------------------------------------------------------------------------
%	HYPERREFERENCES
%----------------------------------------------------------------------------------------

\PassOptionsToPackage{pdftex,hyperfootnotes=false,pdfpagelabels}{hyperref}
\usepackage{hyperref}  % backref linktocpage pagebackref
\pdfcompresslevel=9
\pdfadjustspacing=1

\hypersetup{
% Uncomment the line below to remove all links (to references, figures, tables, etc)
%draft, %JLJL
colorlinks=false, linktocpage=true, pdfstartpage=3, pdfstartview=FitV,
% Uncomment the line below if you want to have black links (e.g. for printing black and white)
%colorlinks=false, linktocpage=false, pdfborder={0 0 0}, pdfstartpage=3, pdfstartview=FitV, 
breaklinks=true, pdfpagemode=UseNone, pageanchor=true, pdfpagemode=UseOutlines,
plainpages=false, bookmarksnumbered, bookmarksopen=true, bookmarksopenlevel=1,
hypertexnames=true, pdfhighlight=/O, urlcolor=webbrown, linkcolor=RoyalBlue, citecolor=webgreen,
%------------------------------------------------
% PDF file meta-information
pdftitle={\myTitle},
pdfauthor={\textcopyright\ \myName, \myUni},
pdfsubject={},
pdfkeywords={},
pdfcreator={pdfLaTeX},
pdfproducer={LaTeX with hyperref and classicthesis}
%------------------------------------------------
}   

%----------------------------------------------------------------------------------------
%	BACKREFERENCES
%----------------------------------------------------------------------------------------



\usepackage{ifthen} % Allows the user of the \ifthenelse command
\newboolean{enable-backrefs} % Variable to enable backrefs in the bibliography
\setboolean{enable-backrefs}{true} % Variable value: true or false

\newcommand{\backrefnotcitedstring}{\relax} % (Not cited.)
\newcommand{\backrefcitedsinglestring}[1]{(Cited on page~#1.)}
\newcommand{\backrefcitedmultistring}[1]{(Cited on pages~#1.)}
\ifthenelse{\boolean{enable-backrefs}} % If backrefs were enabled
{
\PassOptionsToPackage{hyperpageref}{backref}
\usepackage{backref} % to be loaded after hyperref package 
\renewcommand{\backreftwosep}{ and~} % separate 2 pages
\renewcommand{\backreflastsep}{, and~} % separate last of longer list
\renewcommand*{\backref}[1]{}  % disable standard
\renewcommand*{\backrefalt}[4]{% detailed backref
\ifcase #1 
\backrefnotcitedstring
\or
\backrefcitedsinglestring{#2}
\else
\backrefcitedmultistring{#2}
\fi}
}{\relax} 

%----------------------------------------------------------------------------------------
%	AUTOREFERENCES SETUP
%	Redefines how references in text are prefaced for different 
%	languages (e.g. "Section 1.2" or "section 1.2")
%----------------------------------------------------------------------------------------

\makeatletter
\@ifpackageloaded{babel}
{
\addto\extrasamerican{
\renewcommand*{\figureautorefname}{Figure}
\renewcommand*{\tableautorefname}{Table}
\renewcommand*{\partautorefname}{Part}
\renewcommand*{\chapterautorefname}{Chapter}
\renewcommand*{\sectionautorefname}{Section}
\renewcommand*{\subsectionautorefname}{Section}
\renewcommand*{\subsubsectionautorefname}{Section}
}
\addto\extrasngerman{
\renewcommand*{\paragraphautorefname}{Absatz}
\renewcommand*{\subparagraphautorefname}{Unterabsatz}
\renewcommand*{\footnoteautorefname}{Fu\"snote}
\renewcommand*{\FancyVerbLineautorefname}{Zeile}
\renewcommand*{\theoremautorefname}{Theorem}
\renewcommand*{\appendixautorefname}{Anhang}
\renewcommand*{\equationautorefname}{Gleichung}
\renewcommand*{\itemautorefname}{Punkt}
}
\providecommand{\subfigureautorefname}{\figureautorefname} % Fix to getting autorefs for subfigures right
}{\relax}
\makeatother

%----------------------------------------------------------------------------------------

\usepackage{classicthesis} 

%----------------------------------------------------------------------------------------
%	CHANGING TEXT AREA 
%----------------------------------------------------------------------------------------

%\linespread{1.05} % a bit more for Palatino
%\areaset[current]{312pt}{761pt} % 686 (factor 2.2) + 33 head + 42 head \the\footskip



\areaset[current]{342pt}{561pt} % 686 (factor 2.2) + 33 head + 42 head \the\footskip

%for a 6x9 printed format
%\areaset[current]{3.5in}{8in} 

\setlength{\marginparwidth}{7em}%
\setlength{\marginparsep}{2em}%

%----------------------------------------------------------------------------------------
%	USING DIFFERENT FONTS
%----------------------------------------------------------------------------------------

%\usepackage[oldstylenums]{kpfonts} % oldstyle notextcomp
%\usepackage[osf]{libertine}
%\usepackage{hfoldsty} % Computer Modern with osf
%\usepackage[light,condensed,math]{iwona}

%\renewcommand{\sfdefault}{iwona}
%\usepackage{lmodern} % <-- no osf support :-(
%\usepackage[urw-garamond]{mathdesign} <-- no osf support :-(


%\usepackage[sc]{mathpazo} % Use the Palatino font

%\usepackage[T1]{fontenc} % Use 8-bit encoding that has 256 glyphs
%\linespread{1.05} % Line spacing - Palatino needs more space between lines
%\usepackage{microtype} % Slightly tweak font spacing for aesthetics

%
% Font
%
% cmr computer modern roman
\renewcommand{\rmdefault}{cmr}
%\renewcommand{\rmdefault}{sc}

%%%%------------------------------------

% Page length, margin
%\setlength{\topmargin}{-1cm}
%\setlength{\textheight}{240mm}

%%\setlength{\textheight}{240mm}
%%\setlength{\footskip}{1cm}




%%%%%%%%%%%%%%%%%%%%%%%%%%%%%%%%%%%%%%%%%%%%%%%%%%%%%%%%%%%%%%%%%%%%%%
% Fancy-stuff
%%%%%%%%%%%%%%%%%%%%%%%%%%%%%%%%%%%%%%%%%%%%%%%%%%%%%%%%%%%%%%%%%%%%%%

%\usepackage{fancyhdr}
%\pagestyle{fancy} 
%
%% Dutch style of paragraph formatting, i.e. no indents. 
\setlength{\parskip}{1.3ex plus 0.2ex minus 0.2ex}
\setlength{\parindent}{0pt}
%
%
%
%%\newcommand{\myref}[1]{(\ref{#1})}
%%\renewcommand{\captionfont}{\em}
%%\renewcommand{\captionlabelfont}{\em}
%
%%\pagestyle{myheadings}
%%\markboth{}
%
%\fancyhead{}  % clear all header fields
%\fancyfoot{} % clear all footer fields
%



%this is moved to the abstract section
%\renewcommand{\sectionmark}[1]{\markright{\thesection.\ #1}}
%\rhead{\nouppercase{\rightmark}}
%\fancyhead[LE]{  Stories in Immersive Technologies }
%\fancyhead[LO]{ }
%\fancyhead[RO]{ \rightmark}
%\renewcommand{\headrulewidth}{0.4pt}
%\fancyfoot[CO,CE]{\thepage}

%%\renewcommand{\headrulewidth}{0pt}
%%\renewcommand{\footrulewidth}{0pt}







%%%%%%%%%%%%%%%%%%%%%%%%%%%%%%%%%%%%%%%%%%%%%%%%%%%%%%%%%%%%%%%%%%%%%%



%\renewenvironment{quote}
%  {\list{}{\rightmargin\leftmargin}%
%   \item\relax\small}
%  {\endlist}

%for a quote on the uper right side
\let\origquote\quote
%\renewcommand*\quote{\origquote\leftmargin15em}
\renewenvironment{origquote}
 {\slshape
  \list{}{ \leftmargin12em}
   \item\relax }
  {\endlist}





%
% Correction help
%
%\usepackage{showkeys}
%\usepackage{showlabels}


%
% Example theorem definition
%
%\newtheorem{example}{Example}


%
%
% Command definitions
%
%


% Color definitions
\newcommand{\foreigncolor}{MidnightBlue}
\newcommand{\acronymcolor}{\foreigncolor}
\newcommand{\todocolor}{Red}
\newcommand{\suggestioncolor}{Green}
\newcommand{\codecolor}{Black}
\newcommand{\codefilecolor}{\codecolor}
\newcommand{\codetypecolor}{\codecolor}
\newcommand{\codeconfigcolor}{\codecolor}
\newcommand{\codecmdcolor}{\codecolor}


%\todo - This is what needs to be written
\newcommand{\todo}[1]{\textcolor[named]{\todocolor}{\textit{#1}}}
%\newcommand{\todo}[1]{}
%\suggestion - This are ideas to improve
\newcommand{\suggestion}[1]{\textcolor[named]{\suggestioncolor}{
\textit{#1}}}

%\newcommand{\suggestion}[1]{}


%\foreign - Mark foreign words in text
\newcommand{\foreign}[1]{\textcolor[named]{\foreigncolor}{\textit{#1}}}


%\acronym - Special formatting for acronyms and auto indexing
\def\MATLAB{MATLAB\textsuperscript{\textregistered}}


%\code - Code names
\newcommand{\code}[1]{\textcolor[named]{\codecolor}{\texttt{#1}}}
\newcommand{\codefile}[1]{\textcolor[named]{\codefilecolor}{\bf{\texttt{#1}}}}
\newcommand{\codetype}[1]{\textcolor[named]{\codetypecolor}{\bf{\texttt{#1}}}}
\newcommand{\codeconfig}[1]{\textcolor[named]{\codeconfigcolor}{\texttt{#1}}}
\newcommand{\codecmd}[1]{\textcolor[named]{\codecmdcolor}{\texttt{#1}}}


%\secref - Reference to sections
\newcommand{\secref}[1]{\S\ref{#1}}


%
% Vector notation
%
% Unitary vector formatting
\newcommand{\uvec}[1]{\mathbf{\hat{#1}}}
% Complex magnitudes formatting
\newcommand{\cplx}[1]{\tilde{#1}}
% Vector element formatting
\newcommand{\VEC}[1]{\mathbf{#1}}
% Units formatting
\newcommand{\units}[1]{\mathrm{#1}}
% Number with units in math format
\newcommand{\unum}[2]{$#1~\units{#2}$}

